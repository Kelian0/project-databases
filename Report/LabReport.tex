\documentclass[11pt]{article}
\usepackage{graphicx} 
\usepackage{lipsum}
\usepackage{mathrsfs}
\usepackage{amsmath}
\usepackage{{booktabs}}
\usepackage[dvipsnames]{xcolor}
\usepackage{geometry}
\usepackage{indentfirst} 
\usepackage{amssymb}
\usepackage{caption}
\usepackage[
  backend=biber,
  style=apa,
  ]{biblatex}
\addbibresource{MasterDataScience.bib}
\usepackage{hyperref}
  
%--------------------New Command--------------------
  

%--------------------Title Information--------------------
\title{
\includegraphics[height=2cm]{logo/logo_lille.png}\\
\vspace{2cm}
\includegraphics[height=2cm]{logo/Logo-EcoleCentrale-couleur-RVB.png}\\
\vspace{2cm}
\textbf{Report Project Databases: \\Game Industry}\\}
\author{Kélian PONS, Clément MARTIN
		\\
		1\textsuperscript{st} year Master students, Master Data Science\\
        \textbf{Université de Lille}
       } \date{\today}


%--------------------Margin & Indentation--------------------
\geometry{hmargin=1in,vmargin=1in}
\setlength{\parindent}{0.25in}


%--------------------Title Page--------------------
\begin{document}
\maketitle
\thispagestyle{empty}
\newpage
\pagenumbering{roman}

%--------------------Tables of content--------------------
\setcounter{tocdepth}{2}
\tableofcontents
\newpage
% \listoffigures
% \newpage
% \listoftables
% \newpage
% \listofalgorithms
% \newpage
\pagenumbering{arabic}

%--------------------Begining of the Report--------------------


\addcontentsline{toc}{section}{Introduction}   
\section*{Introduction}

The aim of the project for the Databases course is to create a database from a chosen data set.
For this project we choose to create a database for the video game industry, with information on
the games, the developers, the country and town of their headquarters.

\section{Data of the project}

We started by downloading data from different sources:
\begin{itemize}
  \item Video game studios: 
  \href{https://www.kaggle.com/datasets/andreshg/videogamescompaniesregions}{From Kaggle, download here}
  \item Games:
  \href{https://huggingface.co/datasets/FronkonGames/steam-games-dataset}{From Huggingface, download here}
  \item World cities: 
  \href{https://www.kaggle.com/datasets/juanmah/world-cities}{From Kaggle, download here}
  \item World countries:
  \href{https://github.com/bnokoro/Data-Science/blob/master/countries\%20of\%20the\%20world.csv}{From Github, download here}
\end{itemize}


\section{Modelling the database}

\subsection{Video game studios dataset}

The video game studios dataset is a dataset from Kaggle, which contains information on the video game studios.
The csv file contains the following atributes:
\begin{itemize}
  \itemsep0pt
  \item \textbf{Developer}: name of the video game studio
  \item \textbf{City}: city where the video game studio is located
  \item \textbf{Administrative division}: administrative division of the city where the video game studio is located
  \item \textbf{Country}: country where the video game studio is located
  \item \textbf{Est.}: year when the video game studio was founded
  \item \textbf{Notable games, series or franchises}: list of notable games, series or franchises of the video game studio
  \item \textbf{Notes}: notes about the video game studio
\end{itemize}

\subsection{Games dataset}

The games dataset is a dataset from Huggingface, which contains information on the games.
The csv file contains 40 atributes : AppID, Name, Release date, Estimated owners, 
Peak CCU, Required age, Price, Discount, DLCcount, About the game, 
Supported languages, Full audio languages, Reviews, Header image, Website, 
Support url, Support email, Windows, Mac, Linux, Metacritic score, Metacritic url, 
User score, Positive, Negative, Score rank, Achievements, Recommendations, Notes, 
Average playtime forever, Average playtime two weeks, Median playtime forever, 
Median playtime two weeks, Developers, Publishers, Categories, Genres, Tags, Screenshots, Movies,
\\

We only kept the following atributes:
\begin{itemize}
  \itemsep0pt
  \item \textbf{AppID}: The ID of the game.
  \item \textbf{Name}: The name of the game.
  \item \textbf{Release date}: The release date of the game.
  \item \textbf{Estimated owners}: The estimated range of owners of the game.
  \item \textbf{Required age}: The required age to play the game.
  \item \textbf{Price}: The price of the game.
  \item \textbf{DLCcount}: The number of DLCs in the game.
  \item \textbf{Supported languages}: The languages supported by the game.
  \item \textbf{Windows}: True if the game is supported on Windows, False otherwise.
  \item \textbf{Mac}: True if the game is supported on Mac, False otherwise.
  \item \textbf{Linux}: True if the game is supported on Linux, False otherwise.
  \item \textbf{Metacritic score}: The metacritic score of the game.
  \item \textbf{User score}: The user score of the game.
  \item \textbf{Positive}: The number of positive reviews of the game.
  \item \textbf{Negative}: The number of negative reviews of the game.
  \item \textbf{Achievements}: Number of achievements in the game.
  \item \textbf{Average playtime forever}: The average playtime of the game.
  \item \textbf{Developers}: The developers of the game.
  \item \textbf{Categories}: The categories of the game.
  \item \textbf{Genres}: The genres of the game.
\end{itemize}

\subsection{World cities dataset}

The world cities dataset is a dataset from Kaggle, which contains information on the world cities.
The csv file contains the following atributes:
\begin{itemize}
  \itemsep0pt
  \item \textbf{id}: id of the city.
  \item \textbf{City}: name of the city.
  \item \textbf{City ASCII}: name of the city in ASCII.
  \item \textbf{Latitude}: latitude of the city.
  \item \textbf{Longitude}: longitude of the city.
  \item \textbf{Country}: country of the city.
  \item \textbf{ISO2}: ISO2 code of the country.
  \item \textbf{ISO3}: ISO3 code of the country.
  \item \textbf{Admin name}: administrative name of the city.
  \item \textbf{Capital}: Information on the city as a capital.
  \item \textbf{Population}: population of the city.
\end{itemize}


\subsection{World countries dataset}

The world countries dataset is a dataset from Kaggle, which contains information on the world countries.
The csv file contains the following atributes:
\begin{itemize}
  \itemsep0pt
\item \textbf{Country}: name of the country.
\item \textbf{Region}: region of the country.
\item \textbf{Population}: population of the country.
\item \textbf{Area (sq. mi.)}: area of the country.
\item \textbf{Pop. Density (per sq. mi.)}: population density of the country.
\item \textbf{Coastline (coast/area ratio)}: coastline of the country.
\item \textbf{Net migration}: net migration of the country.
\item \textbf{Infant mortality (per 1000 births)}: infant mortality of the country.
\item \textbf{GDP (\$ per capita)}: GDP of the country.
\item \textbf{Literacy (\%)}: literacy of the country.
\item \textbf{Phones (per 1000)}: phones of the country.
\item \textbf{Arable (\%)}: percentage of arable land of the country.
\item \textbf{Crops (\%)}: percentage of crops land of the country.
\item \textbf{Other (\%)}: percentage of other land of the country.
\item \textbf{Climate}: climate of the country.
\item \textbf{Birthrate}: birthrate of the country.
\item \textbf{Deathrate}: deathrate of the country.
\item \textbf{Agriculture}: some agriculture index of the country.
\item \textbf{Industry}: some industry index of the country.
\item \textbf{Service}: some service index of the country.
\end{itemize}

\subsection{Modeling the database}

\begin{figure}[ht]
\centering
\captionsetup{labelfont=bf,labelformat=empty}
\caption{Database Figure}
\includegraphics[width=\textwidth]{figs/Database Diagramm.png}
\end{figure}

After cleaning the data (see next section) we can now organize the tables. 
We choose to have the following tables with the primary keys in red and the foreign keys underlined:
\begin{itemize}
  \itemsep0pt
  \item \textbf{countries}(\textcolor{red}{countryid}, country, region, population, area, pop\_density, coastline, 
  infant\_mortality, phones, arable, crops, other, birthrate, deathrate, agriculture, industry, service)
  \item \textbf{cities}(\textcolor{red}{id}, city, city\_ascii, lat, lng, \underline{countryid}, iso2, iso3, admin\_name, capital, population)
  \item \textbf{studios}(\textcolor{red}{studioid}, name, notable\_games, notes, \underline{cityid}, \underline{countryid}, year)
  \item \textbf{games}(\textcolor{red}{appid}, name, \underline{studioid}, release\_date, required\_age, price, dlccount, windows, mac,
  linux, achievements, estimated\_owners, metacritic\_score, positive, negative, average\_playtime\_forever)
  \item \textbf{categories}(name, \textcolor{red}{categoryid})
  \item \textbf{genres}(name, \textcolor{red}{genreid})
  \item \textbf{languages}(name, \textcolor{red}{languageid})
  \item \textbf{game\_genres}(\textcolor{red}{\underline{appid}, \underline{genreid}})
  \item \textbf{game\_categories}(\textcolor{red}{\underline{appid}, \underline{categoryid}})
  \item \textbf{game\_languages}(\textcolor{red}{\underline{appid}, \underline{languageid}})
\end{itemize}

\section{Population of tables}


\subsection{Modifications for Games}

Since game can have multiple \textbf{genres}, the genres were split into additional tables
(and csv files).
We first create a table containing the \textbf{genreid} and \textbf{name} of the genres. 
Then we create another table containing the \textbf{appid} and the \textbf{genreid}
, since the game can have multiple genres one AppID can 
appear in multiple rows. Finally we remove the original genres column from the games.
\\

The same method was used for the categories and languages supported atributes. So we 
ended up with 6 new tables.
\\

To handle the link with the studios table we modified the \textbf{Developers} column to
match the \textbf{studioid}. For the Developers name that was not in the studios table 
we decided to drop the rows. Since we drop the missmatched data we can set the 
\textbf{studioid} column to not null, in orther words every game must have a studio that is
in the studios table.
\\
\\
The games table contains 1670 games. \\
The genres table contains 28 genres. \\
The categories table contains 29 categories. \\
The languages table contains 121 languages. 


\subsection{Modifications for Studios}

Frist manual cleaning (87 lines) was done on video-games-developers.csv to match the other datasets.
Especially the \textbf{City} and \textbf{Country} columns, which contain non standard names. 
You can find all the manual modification in the modif\_video\_game\_studio.json file.

\begin{figure}[ht]
\centering
\captionsetup{labelfont=bf,labelformat=empty}
\caption{Example of missmatched data between studios and cities datasets}
\includegraphics[width=\textwidth]{figs/Capture d'écran 2025-12-06 081716.png}
\caption{Video game studios dataset}
\includegraphics[width=\textwidth]{figs/Capture d'écran 2025-12-06 081738.png}
\caption{World cities dataset}
\end{figure}

Then a DeveloperID base on the Developer column was created to be the primary key of the table. 
The City column was modified to \textbf{cityid} to match the cities table and be a foreign key.
The Country column was also modified to \textbf{countryid} to match the countries table
and it will be the foreign key to the countries table.
\\

The final studios table contains 686 studios.

\subsection{Modifications for Cities}

For this dataset we had some missmatched and missing data for the countries. We decided 
to handle manually only the countries that were also present in the studios table (only 
the Czech Republic). We dropped the others and modified the country column to 
\textbf{countryid} to be the foreign key to the countries table. 
\\

The final cities table contains 47453 cities.


\subsection{Modifications for Countries}

For this dataset the only modification was to add a new column called 
\textbf{countryid} to be the primary key of the table. Some data formatting was also done
to match the sql formats. 
\\

The final countries table contains 227 countries.

\subsection{Table Creation}

We started by creating the table for the countries since it does not have any foreign keys.
We then add the cities table and the studios table with the constraints of 
the foreign keys. Then we added the games table, the genres table, the categories table
and the languages table. We finished with the game\_genres table, game\_categories table.
\\

We alter some of the table because we put too restrictive constriants on some atributes. 
We then add the data using the csv files.
To see the table creation in detail you can see the \textbf{table\_creation.py} file.

\section{Querying}


\noindent
\begin{minipage}{\linewidth}
\textbf{Free Games}
\begin{verbatim}
SELECT name
FROM games
WHERE price = 0
LIMIT 10;
\end{verbatim}

\begin{tabular}{l}
\toprule
name \\
\midrule
Apex Legends \\
The Lab \\
Back to the Future: The Game \\
Warhammer: Vermintide VR - Hero Trials \\
DRAGON QUEST XI: Echoes of an Elusive Age - Digital Edition of Light \\
STAR WARS: The Old Republic \\
BRINK \\
Prime World: Defenders 2 \\
Resident Evil: Operation Raccoon City \\
Darkfall Unholy Wars \\
\bottomrule
\end{tabular}
\vspace{5ex}

This allows us to identify titles accessible to players without any cost.

\end{minipage}

\noindent
\begin{minipage}{\linewidth}
\vspace{10ex}
\noindent\textbf{Games sorted by price }

\begin{verbatim}
SELECT name, price
FROM games
ORDER BY price DESC
LIMIT 10;
\end{verbatim}

\begin{tabular}{lr}
\toprule
name & price \\
\midrule
Call of Duty: Ghosts - Digital Hardened Edition & 99.990000 \\
Microsoft Flight Simulator 2024 & 69.990000 \\
F1 23 & 69.990000 \\
Suicide Squad: Kill the Justice League & 69.990000 \\
FINAL FANTASY VII REMAKE INTERGRADE & 69.990000 \\
NBA 2K25 & 69.990000 \\
F1 24 & 69.990000 \\
Need for Speed Unbound & 69.990000 \\
Starfield & 69.990000 \\
Indiana Jones and the Great Circle & 69.990000 \\
\bottomrule
\end{tabular}
\vspace{5ex}

We want to list all games sorted by price, as price is an important factor when choosing a video game.
\end{minipage}

\noindent
\begin{minipage}{\linewidth}
\noindent\textbf{Number of games by genre}

\begin{verbatim}
SELECT ge.name, COUNT(*) AS "nb of game"
FROM genres ge
JOIN game_genres gg ON ge.genreID = gg.genreID
GROUP BY ge.genreID
ORDER BY "nb of game" DESC
LIMIT 10;
\end{verbatim}

\begin{tabular}{lr}
\toprule
name & nb of game \\
\midrule
Action & 827 \\
Adventure & 572 \\
RPG & 373 \\
Strategy & 316 \\
Indie & 277 \\
Simulation & 273 \\
Casual & 190 \\
Sports & 109 \\
Racing & 97 \\
Free to Play & 73 \\
\bottomrule
\end{tabular}
\vspace{5ex}

It is interesting to determine which genres are the most developed by studios worldwide. \\
\end{minipage}

\noindent
\begin{minipage}{\linewidth}

\noindent\textbf{All game with a metacritic score after 2020 }

\begin{verbatim}
SELECT name, release_date, metacritic_score
FROM games
WHERE release_date >= '2020-01-01'
AND metacritic_score > 0
ORDER BY metacritic_score DESC
LIMIT 10;
\end{verbatim}

\begin{tabular}{llr}
\toprule
name & release\_date & metacritic\_score \\
\midrule
Mass Effect 2 (2010) Edition & 2023-05-15 & 94 \\
Half-Life: Alyx & 2020-03-23 & 93 \\
Mission: It's Complicated & 2020-02-14 & 91 \\
Crusader Kings III & 2020-09-01 & 91 \\
Microsoft Flight Simulator Game of the Year Edition & 2020-04-17 & 91 \\
The Making of Karateka & 2023-08-29 & 90 \\
Psychonauts 2 & 2021-08-24 & 89 \\
Mass Effect 3 N7 Digital Deluxe Edition (2012) & 2020-06-11 & 89 \\
Apex Legends & 2020-11-04 & 88 \\
DOOM Eternal & 2020-03-19 & 88 \\
\bottomrule
\end{tabular}

\vspace{5ex}

We want all games which have a metacritic score after 2020 since 2020 marks the beginning of the COVID-19 pandemic and the video game’ s bargains exploded.
\end{minipage}

\noindent
\begin{minipage}{\linewidth}
\noindent\textbf{Number of studios by country}

\begin{verbatim}
SELECT c.country, COUNT(*) AS nb_stud
FROM studios s JOIN countries c ON s.countryid = c.countryid
GROUP BY c.country
ORDER BY nb_stud DESC
LIMIT 10;
\end{verbatim}

\begin{tabular}{lr}
\toprule
country & nb\_stud \\
\midrule
United States & 232 \\
Japan & 150 \\
United Kingdom & 73 \\
France & 24 \\
Canada & 21 \\
Germany & 19 \\
Sweden & 17 \\
Korea, South & 14 \\
Poland & 13 \\
Australia & 11 \\
\bottomrule
\end{tabular}

\vspace{5ex}

This allows us to analyze global industry distribution and understand which countries contribute the most in the video game world. \\
\end{minipage}

\noindent
\begin{minipage}{\linewidth}
\noindent\textbf{Biggest Cities with at least 1 studio for each country}

\begin{verbatim}
SELECT co.country, c.city, c.population
    FROM studios s JOIN cities c ON s.cityid = c.id
    JOIN (SELECT MAX(c.population) AS max_pop, c.countryid
          FROM studios s JOIN cities c ON s.cityid = c.id
          GROUP BY c.countryid) AS table_max_pop
    ON table_max_pop.countryid = s.countryid 
    AND table_max_pop.max_pop = c.population
    JOIN countries co ON co.countryid = c.countryid
    GROUP BY co.country, c.city, c.population
    ORDER BY population DESC;
\end{verbatim}

\begin{tabular}{llr}
\toprule
country & city & population \\
\midrule
Japan & Tokyo & 37785000 \\
China & Guangzhou & 26940000 \\
Philippines & Manila & 24922000 \\
Korea, South & Seoul & 23016000 \\
Mexico & Mexico City & 21804000 \\
United States & New York & 18832416 \\
Russia & Moscow & 17332000 \\
Argentina & Buenos Aires & 16710000 \\
India & Bangalore & 15386000 \\
Turkey & Istanbul & 14441000 \\
United Kingdom & London & 11262000 \\
France & Paris & 11060000 \\
Malaysia & Kuala Lumpur & 8911000 \\
Vietnam & Hanoi & 8587100 \\
South Africa & Johannesburg & 7860781 \\
Chile & Santiago & 7171000 \\
Spain & Madrid & 6211000 \\
Australia & Melbourne & 5031195 \\
Germany & Berlin & 4679500 \\
Canada & Montréal & 3675219 \\
Greece & Athens & 3059764 \\
Ukraine & Kyiv & 2952301 \\
Italy & Rome & 2748109 \\
Taiwan & Taipei & 2494813 \\
Romania & Bucharest & 2412530 \\
\bottomrule
\end{tabular}


\vspace{5ex}

It provides insight into where the game industry is concentrated inside each country. \\

It shows whether the largest urban centers are also the most active in game development.
\end{minipage}



\noindent
\begin{minipage}{\linewidth}
\noindent\textbf{Games with the Polish languages supported and the category Multi-player and that are not the Adventure game}

\begin{verbatim}
SELECT g.name
FROM games g
JOIN game_languages gl ON g.appid = gl.appid
JOIN languages l ON l.languageid = gl.languageid
JOIN game_categories gc ON g.appid = gc.appid
JOIN categories c ON c.categoryid = gc.categoryid
WHERE l.name = 'Polish' AND c.name = 'Multi-player'
EXCEPT
SELECT g.name
FROM games g
JOIN game_genres gg ON g.appid = gg.appid
JOIN genres ge ON ge.genreid = gg.genreid
WHERE ge.name = 'Adventure'
ORDER BY name
LIMIT 10;
\end{verbatim}

\begin{tabular}{l}
\toprule
name \\
\midrule
4th \& Inches \\
8-Bit Armies \\
8-Bit Hordes \\
8-Bit Invaders! \\
9-Bit Armies: A Bit Too Far \\
ACL Pro Cornhole \\
Act of Aggression - Reboot Edition \\
Act of War: Direct Action \\
Act of War: High Treason \\
Actua Golf \\
\bottomrule
\end{tabular}

\vspace{5ex}

It allows us to study how many games meet very specific combined criteria which is useful to compare between different genres and categories.

\end{minipage}


\noindent
\begin{minipage}{\linewidth}
\noindent\textbf{The most played game for each genres}

\begin{verbatim}
SELECT ge.name AS "Game Genres", MAX(g.name) AS "The most played game", 
       MAX(g.average_playtime_forever) AS "Time played"
FROM games g
JOIN game_genres as gg ON gg.appid = g.appid
JOIN genres as ge ON ge.genreid = gg.genreid
JOIN (SELECT gg.genreid AS genreid, MAX(g.average_playtime_forever) as max_time_played
      FROM games g
      JOIN game_genres as gg ON gg.appid = g.appid
      GROUP BY gg.genreid) as table_genre_maxtime
      ON table_genre_maxtime.genreid = gg.genreid 
      AND table_genre_maxtime.max_time_played = g.average_playtime_forever
GROUP BY ge.name
ORDER BY "Time played" DESC;
\end{verbatim}

\begin{tabular}{llr}
\toprule
Game Genres & The most played game & Time played \\
\midrule
Strategy & Dota 2 & 37162 \\
Action & Dota 2 & 37162 \\
Free to Play & Dota 2 & 37162 \\
Massively Multiplayer & FINAL FANTASY XIV Online & 27478 \\
RPG & FINAL FANTASY XIV Online & 27478 \\
Casual & MARVEL Puzzle Quest & 18149 \\
Adventure & Rust & 16623 \\
Indie & Rust & 16623 \\
Simulation & Arma 3 & 12276 \\
Sports & NBA 2K20 & 9237 \\
Racing & Micro Machines World Series & 7004 \\
Early Access & Mount \& Blade II: Bannerlord & 5482 \\
Video Production & Source Filmmaker & 1846 \\
Animation \& Modeling & Source Filmmaker & 1846 \\
Violent & Pathfinder Adventures & 649 \\
Sexual Content & Mary Skelter: Nightmares & 400 \\
Nudity & X-Blades & 291 \\
Web Publishing & Multiplicity & 2 \\
Audio Production & Multiplicity & 2 \\
Design \& Illustration & Multiplicity & 2 \\
Education & Multiplicity & 2 \\
Photo Editing & Multiplicity & 2 \\
Software Training & Multiplicity & 2 \\
Utilities & Multiplicity & 2 \\
Accounting & Multiplicity & 2 \\
Game Development & Starfield: Creation Kit & 0 \\
Gore & Warhammer: Vermintide VR - Hero Trials & 0 \\
Free To Play & Project Fireball & 0 \\
\bottomrule
\end{tabular}

\vspace{5ex}

We retrieve the most played game for each genre in order to help the customer to find the best option within their preferred genres. \\
\end{minipage}

\noindent
\begin{minipage}{\linewidth}
\noindent\textbf{Top 3 games in 2020 which has at least 3 supported languages}

\begin{verbatim}
SELECT ga.name as "Game name", ga.metacritic_score as "Metacritic score",
       COUNT(gl.languageID) AS "language count"
FROM games ga
JOIN game_languages AS gl USING (appid)
WHERE ga.release_date >= '2020-01-01' AND ga.release_date < '2021-01-01'
GROUP BY ga.appid
HAVING COUNT(gl.languageID) >= 3
ORDER BY ga.metacritic_score DESC
LIMIT 3;
\end{verbatim}

\begin{tabular}{lrr}
\toprule
Game name & Metacritic score & language count \\
\midrule
Half-Life: Alyx & 93 & 12 \\
Microsoft Flight Simulator Game of the Year Edition & 91 & 12 \\
Crusader Kings III & 91 & 12 \\
\bottomrule
\end{tabular}

\vspace{5ex}

We retrieve the top 3 games of 2020 that support at least three languages, since global accessibility became especially important during the pandemic\\
\end{minipage}

\noindent
\begin{minipage}{\linewidth}
\noindent\textbf{Average games prices for each studio in Tokyo or New York or Montréal with more than 3 games}

\begin{verbatim}
SELECT s.name as "studio name", AVG(ga.price) as "average price", 
       COUNT(ga.name) as "nb of game",ci.city as city , ci.iso3 as country
FROM games ga
JOIN studios AS s USING(studioid)
JOIN cities AS ci ON ci.id = s.cityid
GROUP BY s.studioid, s.name, ci.iso3, ci.city
HAVING (ci.city = 'Tokyo' OR ci.city = 'New York' OR ci.city = 'Montréal') 
       AND COUNT(ga.name) >= 3
ORDER BY "average price"
LIMIT 20;
\end{verbatim}

\begin{tabular}{lrrll}
\toprule
studio name & average price & nb of game & city & country \\
\midrule
Rockstar Games & 14.578000 & 10 & New York & USA \\
FromSoftware & 19.996667 & 3 & Tokyo & JPN \\
Idea Factory & 22.767778 & 9 & Tokyo & JPN \\
Compulsion Games & 25.990000 & 3 & Montréal & CAN \\
Square Enix & 26.680877 & 57 & Tokyo & JPN \\
Tango Gameworks & 29.992000 & 5 & Tokyo & JPN \\
Tamsoft & 32.212222 & 9 & Tokyo & JPN \\
Nihon Falcom & 33.606667 & 30 & Tokyo & JPN \\
Tokyo RPG Factory & 46.656667 & 3 & Tokyo & JPN \\
\bottomrule
\end{tabular}

\vspace{5ex}
This query is interesting because it allows us to compare pricing strategies across different regions of the global video game industry since these cities represent important cultural, economic, and technological centers.
\end{minipage}

\newpage

\section{Database structure update}

We can imagine extending our database so that it no longer includes only games from Steam, which focuses on computer games. Instead, we may want to incorporate games from others \texttt{platforms} , such as the \textit{PlayStation 5} or \textit{Nintendo Switch}. \\

To do this, we need to obtain a data set that lists all available platforms and create a dedicated table called, for example, platform, that stores information about each platform.\\

We then introduce an associative table, \texttt{game\_platform}, which links \texttt{games} to \texttt{platforms}. This relationship is many-to-many, since multiple games can be available on several platforms, and each platform can support many games. \\


To design the platform table, we assign a serial integer as its primary key and include a non-null name attribute. We may also store additional information, such as the release date, price, or total sales, for each platform. \\

Finally, the \texttt{game\_platform table} must contain two attributes that act as foreign keys referencing the \texttt{game} and \texttt{platform} tables. Together, these two attributes should form a composite primary key to ensure that each game–platform pair is unique.



\end{document}